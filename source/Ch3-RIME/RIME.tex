% !TeX spellcheck = en_UK

\chapter{Mathematical Framework: the RIME Formalism}\label{section.RIME}
\minitoc

\pg
In this section, we introduce the mathematical framework which forms the basis for our algorithmic work.
While radio interferometry has historically been a complex business \citepads{2001isra.book.....T}, modern radio interferometry is based on the Radio Interferometer's Measurement Equation, a powerful and elegant formalism which underpins modern calibration and imaging algorithms. In this section, I will give a simplified account of this formalism as described in \citetads{2011A&A...527A.106S} and its companion papers\footnote{\citetads{2011A&A...527A.107S}, \citetads{2011A&A...527A.108S} and \citetads{2011A&A...531A.159S}}. This will form the theoretical basis on which further sections will expand. I cannot overstate the importance of the RIME as the foundation of modern calibration and imaging algorithms.

\pg
For a mathematically rigorous description of the RIME, the references used in this work (particularly \citetads{2011A&A...527A.108S} and \citetads{2011A&A...531A.159S}) are obviously the first place to look. We frame the RIME in the continuity of previous theoretical frameworks of radio interferometry, while also describing it in the context of contemporary software packages and algorithmic tools.

\section{Setting up the RIME: a single point source}
\label{section.RIME.setup}

\pg
The Radio Interferometer's Measurement Equation, or RIME, is a formalism which allows for an elegant and efficient formulation of the physical processes which affect signal propagation, from astrophysical effects to instrumental effects. Its fundamental underlying hypothesis is \emph{linearity}: that transformations along the signal are linear with respect to the basis chosen to represent our signal.

\pg
Consider a single quasi-monochromatic point source in the sky. Its signal at a point in time and space can be described by a complex vector $\evector$. We can then represent physical processes which affect this signal's propagation using \emph{Jones matrices}. In other words:
\begin{equation}
\volt = \Jones\evector
\end{equation}
where $\volt$ is the voltage measured by our antenna, and $\Jones$ the Jones matrix describing the \emph{net} propagation effects - atmospheric, astrophysical, and instrumental - which affect our source's signal. $\Jones$ can be written as a series of matrix products, each individual matrix describing a physical phenomenon. They are then called a \emph{Jones chain}, and the resulting $\Jones$ is often called the \emph{total Jones matrix}.

\pg
A visibility is the correlation of voltage from two antennas. In other words,
\begin{align}
\Vis_{pq} &= 2 \volt_p (\volt_q)^\Herm\\
\Vis_{pq} &= 2 \Jones_p \evector (\Jones_q \evector)^\Herm\\
\Vis_{pq} &= 2 \Jones_p \evector \evector^\Herm \Jones_q^\Herm
\end{align}
where $\Jones_p$ corresponds to the signal propagation chain of antenna $p$, and $\Jones_q$ for antenna $q$. Note that a factor of 2 has been introduced here, as a matter of convention. This is due to the definition of the Stokes parameter in relation to our $\evector$ outer product.

\pg
If we choose to use $xyz$ as our coordinates, with $z$ the direction of propagation of our signal, then \citetads{1996A&AS..117..137H} show that the following relation holds:
\begin{equation}
2 \evector \evector^\Herm = 2  \begin{bmatrix} <e_x e_x^\star> & <e_x e_y^\star> \\ <e_y e_x^\star> & <e_y e_y^\star> \end{bmatrix} = \begin{bmatrix} I+Q & U+iV \\ U-iV & I-Q \end{bmatrix} = \Bmatrix
\end{equation}

\pg
We can thus write:
\begin{equation}
\Vis_{pq} = \Jones_p \Bmatrix \Jones_q^\Herm
\end{equation}

\pg
This gives us an elegant formulation of the relationship between the signal emitted by an astrophysical source and its corresponding measured visibility. The Stokes parameters of our source can be directly calculated from our measured visibility, provided $\Jones_p$ and $\Jones_q$ are known. Of course, in practice, they are not, and must therefore be modelled. To do this, we must expand each Jones matrix into a corresponding Jones chain.

\section{Expanding our Jones Chain}
\label{section.RIME.JonesChain}
\pg
Having written our RIME as a matrix multiplication problem, we can now begin to differentiate between different types of Jones matrices. $\Jones_p$ describes the \emph{aggregate} effects which occur over the course of the signal's propagation to our antenna $p$. It can be split into a Jones chain of $n$ separate matrices, corresponding to different effects:
\begin{align}
\displaystyle \Jones_{sp}^H &= \prod_1^n \Jones_{nsp}^H = \Jones_{1sp}^H \Jones_{2sp}^H ... \Jones_{nsp}^H\\
\displaystyle \Jones_{sp}   &= \Jones_{nsp}...\Jones_{2sp} \Jones_{1sp}
\end{align}

\pg
Since the Jones terms are applied sequentially to the brightness matrix, starting with $\Jones_{1sp}$, the terms to the right of our decomposition of $\Jones_{sp}$ can be said to occur `at the source', i.e. before those to the left (said to occur `at the antenna').

\pg
Our aim is thus to find the Jones chain which most concisely describes the different \emph{types} of effects affecting our signal. In order of increasing complexity, there are three: so-called \emph{geometric} effects, \emph{direction-independent} effects, and \emph{direction-dependent} effects.

\subsection{Geometric Effects: the Phase Delay Jones Matrix}
\label{section.RIME.JonesChain.Kjones}

\pg
The physical quantity measured by an interferometric array, the \emph{visibility}, is not a function of the absolute phase of our signal, but rather a function of the \emph{phase difference} between our measured voltages $\volt_p$ and $\volt_q$. In other words, it is a function of the phase difference on \emph{baseline} $pq$, which consists of antennas $p$ and $q$. The direction which minimises this difference is called the \emph{phase centre}.

\pg
We shall use the conventional coordinate system and notations \citepads[][etc]{2011A&A...527A.106S,2001isra.book.....T}. We thus have a $z$ axis pointing towards the phase centre, and an antenna $p$ at coordinates $\U_p=(u_p,v_p,w_p)$. The phase difference between $p$ and phase centre (where, by definition, $\U=0$) for a signal arriving from direction $\pmb{\sigma}$ is then:
\begin{equation}
k_p=2\pi i (( u_p l + v_p m + w_p (n-1) ))
\end{equation}

where $n=\sqrt{1-l^2-m^2}$. Here, $l,m,n$ are the direction cosines\footnote{The $n-1$ term in the exponential is due to the fact that, at phase centre, $k_p=0$ for all $\U$ and $n=\sqrt{1-0-0}=1$.} of $\pmb{\sigma}$, and $\U$ is defined in units of wavelength\footnote{This means that, for a given baseline $pq$, $\U$ will vary as a function of frequency.}.

\pg 
Having done this, we can now introduce a scalar \emph{K-Jones} matrix to represent the effect of this phase delay. Since it is scalar, it will commute with all other Jones matrices \citepads[see]{2011A&A...527A.106S}. It can be be written as:
\begin{equation}
\Kjones_p = e^{-i k_p}  = e^{-2\pi i ( u_p l + v_p m + w_p (n-1) )}
\end{equation}

\subsection{The Coherency Matrix}
\label{section.RIME.JonesChain.CoherencyMatrix}

\pg
The commutation property of $\Kjones_p$ with other Jones matrices allow us to always place it at the rightmost end of our Jones chain. We can thus write our Jones chain as follows:
\begin{align}
\Jones_p  &= \Gjones_p \Kjones_p\\
\end{align}

\pg
Here, we have decoupled the nominal phase delay term ($\Kjones_p$) from all `corrupting' effects ($\Gjones_p$). We can take this one step further by defining the \emph{source coherency}, $\Coherency_{pq}$, as follows:
\begin{align}\label{eq.def.coherency}
\Coherency_{pq} &= \Kjones_p \Bmatrix_{pq} \Kjones_p^H\\
\Vis_{pq} &= \Gjones_p \Coherency_{pq} \Gjones_q^H
\end{align}

\pg
This is useful on a conceptual level: the process of \emph{calibration} is now to find a best estimate for $\Gjones_p$, which contain all non-`geometric' physical effects on the signal's propagation from an astrophysical source to our voltage measurement.

\pg
On a practical level, using the coherency matrix $\Coherency$ allows us to write our RIME more concisely, by absorbing the geometric effects into the core of our RIME. While not strictly necessary, it can be a useful thing to do.

\section{Multiple Point Sources}
\label{section.RIME.MultiplePointSources}

\pg
So far, $\Bmatrix$ represents the signal from a single point source. However, the formalism can be extended to multiple point sources. This is because electric fields -- and therefore radio signals -- are additive. If our sky contains $s$ point sources, then our RIME becomes:
\begin{equation}
\Vis_{pq} = \sum_s \Jones_{sp} \Bmatrix_s \Jones_{sq}^\Herm
\end{equation}

\pg
We now expand our Jones chain as before. Some, but not all, elements in the chain will only depend on the antenna. These will be the same for all sources. We can thus write our Jones chain as:
\begin{equation}
\Jones_{sp} = G_p E_{sp} K_{sp}
\end{equation}

\pg
Note that we have kept the source-dependent effects (i.e. $E_{sp}$ those occurring `at the source') are to the right of our chain, and the antenna-dependent ($G_p$) effects to the left. $K_{sp}$ is a scalar and can thus be moved at will; it is therefore placed at the far right to recover the coherency matrix if necessary.

\pg
Our multiple-source RIME can thus be written in the following form:
\begin{equation}\label{eq.multiple.source.RIME}
\Vis_{pq} = \Gjones_p \bigg( \sum_s \Ejones_{sp} \Kjones_{sp} \Bmatrix_{s} \Kjones_{sq}^H \Ejones_{sq}^H \bigg) \Gjones_q^H
\end{equation}

\pg
This `onion' form of the RIME is usually sufficient to describe signal propagation in radio-interferometric arrays. We find the three types of Jones matrices promised earlier: the geometric effects accounted for by $\Kjones$, the direction-dependent effects modeled into $\Ejones$, and the direction-independent effects (antenna-based effects) modeled into $\Gjones$.

\pg
There are still limitations to this RIME; notably, it assumes that the sky consists entirely of point sources (absence of diffuse emission), and that Jones matrices are time-independent. These are of course unphysical hypotheses, though they are a very good first approximation. We shall now address these two points.

\section{Diffuse Emission: the Full-sky RIME}
\label{section.RIME.FullSky}

\subsection{Deriving the Fully-Sky RIME}
\label{section.RIME.FullSky.derivation}
\pg
If we want to be able to take diffuse emission properly into account, we must go from our set of discrete point sources $\Bmatrix_s$ and integrate the underlying continuous brightness distribution $\Bmatrix(\direction)$ over all possible directions. This gives us the following expression:
\begin{equation}
\Vis_{pq} = \int_{4\pi} \Jones_p(\direction) \Bmatrix(\direction) \Jones_q^H(\direction)d\Omega
\end{equation}

\pg
Here, effects such as our array's beam lie in the Jones matrices. By performing the analysis shown in Sect. 3.1 of \citetads{2001isra.book.....T}, we can rewrite this expression in terms of a sine projection of the sphere onto the $(l,m)$ plane tangential at field centre. The integral then becomes:
\begin{equation}
\Vis_{pq} = \int_{l,m} \Jones_p(\direction) \Bmatrix(\direction) \Jones_q^H(\direction)\frac{dl dm}{n}
\end{equation}

\pg
Having written this, and using $\lvec$ as shorthand $(l,m)$, we can once again decompose our Jones matrix into a Jones chain containing a direction-independent term $\Gjones$, a direction-dependent term $\Ejones$, and the phase term $\Kjones$. Substituting this into our integral, we get the following expression:
\begin{align}
\Vis_{pq} &= \Gjones_p \bigg(\int_{\lvec} \frac{1}{n} \Ejones_p(\lvec) \Bmatrix \Ejones_q(\lvec) \Kjones_p(\lvec)\Kjones_q^H(\lvec) dl dm \bigg) \Gjones_q^H\\
          &= \Gjones_p \bigg(\int_{\lvec} \frac{1}{n} \Ejones_p(\lvec) \Bmatrix \Ejones_q(\lvec) e^{-2\pi i\big(u_{pq}l+v_{pq}m+w_{pq}(n-1)\big)} dl dm \bigg) \Gjones_q^H
\end{align}
where
\begin{alignat}{3}
u_{pq}&=u_p-u_q \qquad v_{pq}&=v_p-v_q \qquad w_{pq}&=w_p-w_q
\end{alignat}

\pg
This is a three-dimensional Fourier transform. It can be reduced to a two-dimensional Fourier transform by writing the $n$-dependence in the exponential (and normalisation) as a Jones matrix, thus removing the \emph{non-coplanarity} terms from the explicit RIME. We can thus define the following:
\begin{align}
\Wjones_p &= \frac{1}{\sqrt{n}}e^{-2\pi i w_p(n-1)}\\
B_{pq}    &= \Ejones_p \Wjones_p \Bmatrix \Wjones_q^H \Ejones_q^h
\end{align}
and write our RIME in its final form:
\begin{equation}\label{eq.vis.FT.coherency.matrix}
\Vis_{pq} = \Gjones_p \bigg(\int_{\lvec} \Bmatrix_{pq}(\lvec) e^{-2\pi i\big(u_{pq}l+v_{pq}m+w_{pq}(n-1)\big)} dl dm \bigg) \Gjones_q^H
\end{equation}
\pg
Note that $\Bmatrix$ was the actual sky  brightness distribution as a function of direction, but now $\Bmatrix_{pq}$ is the sky brightness \emph{as seen by baseline pq} - there is no \emph{a priori} reason this should be the same for all baselines.\\

\subsection{Recovering the van Cittert-Zernike Theorem}
\label{section.RIME.FullSky.CVZ}

\pg
Earlier calibration models and algorithm rely on the hypothesis that all baselines see the same sky. This is the \emph{fundamental premise of self-calibration}. The topic of self-cal is covered in \cref{section.calibration.2gc}. This premise only holds when all Direction-Dependent Effects (henceforth referred to as DDEs) are identical for all baselines (and therefore for all antennas). This is a necessary condition for the apparent sky $\Bmatrix_{pq}$ to be the same for all baselines\footnote{$\Bmatrix_{pq}$ would then correspond to the `true' sky attenuated by the ``power beam", which corresponds to our instrument's directional sensitivity in the traditional view}, as it allows us to write:
\begin{align}
\Ejones_p(\lvec) \Wjones_p(\lvec)&= \Ejones(\lvec) \Wjones(\lvec)\\
\Bmatrix_{pq}(\lvec) = \Bmatrix_{app}(\lvec) &= \Ejones(\lvec) \Wjones(\lvec) \Bmatrix)(\lvec) \Wjones^H(\lvec) \Ejones^H(\lvec)
\end{align}

\pg
If this condition is met, we can then rewrite the full-sky RIME as:
\begin{equation}\label{eq.fullsky.RIME}
\Vis_{pq} = \Gjones_p \Coherency_{pq} \Gjones_q^H
\end{equation}
where $\Coherency_{pq} = \Coherency(u_{pq},v_{pq}) = \Coherency(\uvec)$. Comparing \cref{eq.def.coherency,eq.vis.FT.coherency.matrix} directly shows that the coherency matrix is simply the two-dimensional Fourier transform of $\Bmatrix_{app}(\lvec)$. We can thus call $\Coherency(\uvec)$ the \emph{sky coherency}, in keeping with our nomenclature for the RIME of a point source.

\pg
Note that deriving the van Cittert-Zernike theorem (\citepads{1934Phy.....1..201V}, covered in \citepads{2001isra.book.....T}) from the RIME was simply of matter of treating phase as a Jones matrix in its own right\footnote{\href{https://raw.githubusercontent.com/wiki/ska-sa/meqtrees/aips++_note185.pdf}{Noordm, J.E. 1996, AIPS++, Note 185: The Measurement Equation of a Generic Radio Telescope}}. A significant consequence of this is that DDEs can be incorporated within the RIME formalism.

\section{Time-dependent Jones matrices, Smearing, and Decoherence}
\label{section.RIME.TimeDep}

\subsection{Time-dependence of Jones matrices}
\label{section.RIME.TimeDep.Jones}

\pg
Until this point, we have not considered any kind of sky variability in time. In effect, \cref{eq.fullsky.RIME} describes the snapshot taken by an interferometer for a single measurement. Limiting ourselves to this scenario, however, strongly limits the sampling of the $uv$-plane which can be performed. Indeed, interferometers generally rely on the Earth's rotation to rotate their baselines $pq$, thus increasing $uv$-coverage at no additional cost.This technique is known as \emph{super-synthesis} (\cite{supersynthesis}\footnote{\href{http://www.jpier.org/PIERB/pierb22/05.10032105.pdf}{(link to paper)}}) For \cref{eq.fullsky.RIME} to hold throughout an observation, we must therefore assume that $B_{app}$ remains constant in time over the course of the observation. In effect, this means assuming that
\begin{equation}\label{eq.selfcal.requirement}
\Ejones_p(t,\lvec) = \Ejones_p(\lvec) = \Ejones(\lvec) \quad \forall (t,p)
\end{equation}

\pg
The above means that direction-dependent effects must be time-independent for \cref{eq.fullsky.RIME} to hold. We can thus split direction-dependent effects into two categories: \emph{trivial} DDEs, which satisfy this condition (and can thus be solved for within the RIME formalism), and \emph{non-trivial} DDEs, which are - as the name indicates - non-trivial to account for within the RIME formalism. An example of a trivial DDE is the primary gain; and example of a non-trivial (i.e. time-variable) DDE is ionospheric turbulence.

\subsection{Smearing and Decoherence}
\label{section.RIME.TimeDep.Decoherence}

\pg
Of course, our Jones matrices are not the only time-dependent element of the RIME formalism. As mentioned above, to image with interferometric arrays, radio astronomers rely on the Earth's rotation to better sample $uv$-space. This means that $\uvec$ becomes a \emph{time-dependent} variable.

\pg
Furthermore, if we choose to define $\uvec$ in terms of wavelength, $\uvec_p$ becomes dependent in frequency. Our time-independent, frequency-independent RIME thus becomes only valid for infinitesimal increments in time and frequency.

\pg
In practice, measurements made by any interferometric array are actually averaged over the integration time of the measurement, as well as the frequency channel size. A better formulation is therefore to explicitly write the RIME as an integration over time and frequency bins:
\begin{align}
\langle \Vis_{pw} \rangle &= \frac{1}{\Delta t \Delta \nu} \int_{t_0}^{t_1} \int_{\nu_0}^{\nu_1} \Vis_{pq}(t,\nu) dv dt\\
                          &= \frac{1}{\Delta t \Delta \nu} \int_{t_0}^{t_1} \int_{\nu_0}^{\nu_1} \Jones_p(t,\nu) \Bmatrix \Jones_q^H(t,\nu)
\end{align}

\pg
$\Jones(t,\nu)$ describes the propagation of an electromagnetic signal through various physical processes. This signal has a complex phase, which is variable in both time and frequency. Since it is a rotating vector, integrating over any time or frequency band will always result in a \emph{net loss} in amplitude in the measured $\langle\Vis\rangle$. This mechanism is commonly referred to as \emph{time/bandwidth width decorrelation} or \emph{smearing} \citepads{2011A&A...527A.106S}.

\pg
The general effect is referred to as \emph{decoherence}, and the specific case of decoherence caused by the $\Kjones$ term is referred to as \emph{smearing}.

\pg Smearing increases with baseline length ($\uvec_{pq}$) and distance from phase centre ($l,m$). The noise amplitude, however, does not decrease. This means that smearing results in a decrease in sensitivity, which becomes significant when attempting to do wide-field, high-resolution images (see \cref{section.EGS.hires}).

\pg 
In the context of the RIME, smearing is an element-by-element operation, i.e. different elements do not affect each other. Treating smearing within the RIME is thus a trivial expansion of the scalar equations.

\pg
Assuming that $\Delta t$ and $\Delta \nu$ are small enough that the amplitude of $\Vis_{pq}$ remains constant over the integration band (i.e. that $\Vis_{pq}$ is well-sampled by our instrument), then a useful first-order approximation is:
\begin{equation}
\langle \Vis_{pw} \rangle \simeq \text{sinc}\frac{\Delta \pmb{\Psi}}{2} \enskip \text{sinc}\frac{\Delta \pmb{\Phi}}{2} \enskip \Vis_{pq}(t_{mid},\nu_{mid})
\end{equation}
where
\begin{align}
t_{mid}           &= (t_0 + t_1)/2\\
\nu_{mid}         &= (\nu_0 + \nu_1)/2
\end{align}
\begin{align}
\Delta \pmb{\Psi} &= \arg \Vis_{pq}(t_1,\nu_{mid}) - \arg \Vis_{pq}(t_0, \nu_{mid})\\
\Delta \pmb{\Phi} &= \arg \Vis_{pq}(t_{mid},\nu_1) - \arg \Vis_{pq}(t_{mid}, \nu_0)
\end{align}

\pg
This equation is convenient to determine the impact of decorrelation for a source at a given distance from phase centre. Note, however, that it can only - by definition - be valid for a single source at a time. If using a RIME such as \cref{eq.multiple.source.RIME}, then it can be calculated for each source individually.

\section{Noise}
\label{section.RIME.noise}

\pg
The RIME presented thus far has been for an ideal case, free of noise in our measured visibility. In practice, each complex visibility will be affected by uncorrelated Gaussian noise\footnote{Detailed treatment given in \citepads{2001isra.book.....T}, section 6.2}, which means that a RIME of the same form as \cref{eq.multiple.source.RIME} would have to be formuled at follows:
\begin{equation}
\Vis_{pq} = \Gjones_p \bigg( \sum_s \Ejones_{sp} \Kjones_{sp} \Bmatrix_{s} \Kjones_{sq}^H \Ejones_{sq}^H \bigg) \Gjones_q + \sigma_{pq}
\end{equation}
where $\sigma_{pq}$ is a $(2,2)$ matrix with real and complex parts in each component. A similar addition can be made for any of the RIMEs presented here.

\pg
This noise sets a hard limit on the sensitivity achieved for a given observation, and `reaching the noise' becomes the gold standard of calibration \citepads{2011A&A...527A.107S}. One must then take $\sigma_{pq}$ into account properly when solving for the Jones matrices.



