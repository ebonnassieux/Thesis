\title{Statistical analysis of noise properties in radio-interferometric images\\ \& Applications to low-frequency extragalactic field radio images}

\author{ETIENNE Bonnassieux}

\supervisor{Cyril Tasse}
% \doctoralschool{Sciences et Métiers de l'Ingénieur}{521}
% \specialty{Mathématiques, Informatique Temps-Réel, Robotique}
\date{20 Septembre 2018}

\jury{
  M Pierre Kervella, Président\\
  LESIA

  M Neal Jackson, Rapporteur\\
  University of Manchester

  M Patrick Charlot, Rapporteur\\
  Laboratoire d'Astrophysique de Bordeaux %, Rapporteur

  Mme Maajke Mevius\\
  ASTRON, Membre du jury

  M Chiara Ferrari, Membre du jury\\
  Observatoire de la Côte d'Azur

  M Cyril Tasse, Directeur de Thèse\\
  GEPI

  M Oleg Smirnov, Directeur de Thèse\\
  SKA-SA

  M Philippe Zarka, Directeur de Thèse\\
  LESIA
}

\frabstract{
  Un nouvel algorithme permettant des améliorations dramatiques dans la création d'images interférométriquesà faible coût computationnel est développé dans l'optique d'être utilisée pour LOFAR et autres précurseurs SKA. Cet algorithme est déployé dans le pipeline LOFAR, qui est utilisé pour créer la première image grand-champ et haute-résolution faite avec le LOFAR international. Cette image, du champ extragalactique Extended Groth Strip, possède une résolution comparable à celle du télescope Hubble.
}

\enabstract{
  A new algorithm was devised to dramatically improve radio interferometric images at near-zero computational cost. This algorithm was deployed in the LOFAR pipeline and used to create a high-resolution, wide-field image of a famous deep extragalactic field, the Extended Groth Strip using international LOFAR. This image will match the Hubble Space Telescope's resolution.
}

\frkeywords{ Caesar licentia post honoratis haec adhibens urbium
  honoratis nullum Caesar.}
\enkeywords{ Delatus delatus nominatus
  onere aut trahebatur quod tenus et bonorum.}