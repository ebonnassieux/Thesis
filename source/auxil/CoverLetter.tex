\title{Statistical Analysis of the Radio-Interferometric Measurement Equation, a derived adaptive weighting scheme, and applications to LOFAR-VLBI observation of the Extended Groth Strip}

\author{ETIENNE Bonnassieux}

\supervisor{Oleg Smirnov}
% \doctoralschool{Sciences et Métiers de l'Ingénieur}{521}
% \specialty{Mathématiques, Informatique Temps-Réel, Robotique}
\date{20 Septembre 2018}

\jury{
  Prof Pierre Kervella, Président\\
  LESIA

  Prof Neal Jackson, Rapporteur\\
  University of Manchester

  Prof Patrick Charlot, Rapporteur\\
  Laboratoire d'Astrophysique de Bordeaux %, Rapporteur

  Dr Maaijke Mevius\\
  ASTRON, Membre du jury

  Dr Chiara Ferrari, Membre du jury\\
  Observatoire de la Côte d'Azur

  Dr Cyril Tasse, Directeur de Thèse\\
  GEPI

  Prof Oleg Smirnov, Directeur de Thèse\\
  SKA-SA

  Dr Philippe Zarka, Directeur de Thèse\\
  LESIA
}

\frabstract{
Grâce à une analyse statistique de l'Equation de la Mesure des Interféromètres Radio, un schéma de pondération adaptatif est dérivé, basé sur la qualité de calibration des données d'un instrument interféromètrique. Ce schéma est utilisé sur une observation d'un champ extragalactique, l'Extended Groth Strip, observation qui contient une source radio-vive (3C295) dans son champ de vue. Cette source est résolue avec LOFAR-VLBI; un modèle de source est créé afin de calibrer les stations LOFAR internationales. Cela permettra d'imager le champ à une résolution comparable à celle du Hubble Space Telescope, dont des données sont disponibles pour ce champ extragalactique.
}

\enabstract{
By performing a statistical analysis of the Radio Interferometer's Measurement Equation, we derive adaptive quality-based weighting schemes. These are deployed on an observation of the Extended Groth Strip, which includes a bright 3C source in the field of view. This source, which is resolved for LOFAR-VLBI, is modeled and used as a calibrator source for the Extended Groth Strip. This will allow the field to be imaged with a resolution matching the Hubble Space Telescope's, of which data are available for this field.
}

\frkeywords{Interférométrie, radio, LOFAR, algorithmie, extragalactique}
\enkeywords{Interferometry, radio, LOFAR, algorithms, extragalactic}