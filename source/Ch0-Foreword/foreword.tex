\chapter*{Foreword}

\pg
J.R.R Tolkien wrote, in his Mythopoeia, that ``He sees no stars who does not see them first, of living silver made that sudden burst, to flame like flowers beneath the ancient song"\footnote{See \href{http://home.agh.edu.pl/~evermind/jrrtolkien/mythopoeia.htm}{here} for the full text.}. In his defense of myth-making, he formulates the argument that the attribution of meaning is an act of creation - that ``trees are not `trees' until so named and seen" - and that this capacity for creation defines the human creature. The scientific endeavour, in this context, can be understood as a social expression of a fundamental feature of humanity, and from this endeavour flows much understanding. This thesis, one thread among many, focuses on the study of astronomical objects as seen by the radio waves they emit.

\pg
What are radio waves? Electromagnetic waves were theorised by James Clerk Maxwell \citep{maxwell} in his great theoretical contribution to modern physics, their speed matching the speed of light as measured by Ole Christensen R\o mer and, later, James Bradley. It was not until Heinrich Rudolf Hertz's 1887 experiment that these waves were measured in a laboratory, leading to the dawn of radio communications - and, later, radio astronomy. The link between radio waves and light was one of association: light is known to behave as a wave (Young double-slit experiment), with the same propagation speed as electromagnetic radiation. Light ``proper" is also known to exist beyond the optical regime: Herschel's experiment shows that when diffracted through a prism, sunlight warms even those parts of a desk which are not observed to be lit (first evidence of infrared light). The link between optical light and unseen electromagnetic radiation is then an easy step to make, and one confirmed through countless technological applications (e.g. optical fiber to name but one). And as soon as this link is established, a question immediately comes to the mind of the astronomer: what does the sky, our Universe, look like to the radio ``eye"?

\pg
Radio astronomy has a short but storied history: from Karl Jansky's serendipitous observation of the centre of the Milky Way, which outshines our Sun in the radio regime, in 1933, to Grote Reber's hand-built back-yard radio antenna in 1937, which successfully detected radio emission from the Milky Way itself\footnote{\url{https://www.nrao.edu/whatisra/hist_reber.shtml}}, to such monumental projects as the Square Kilometer Array and its multiple pathfinders, it has led to countless discoveries and the opening of a truly new window on the Universe. The work presented in this thesis is a contribution to this discipline - the culmination of three years of study, which is a rather short time to get a firm grasp of radio interferometry both in theory and in practice. The need for robust, automated methods - which are improving daily, thanks to the tireless labour of the scientists in the field - is becoming ever stronger as the SKA approaches, looming large on the horizon; but even today, in the precursor era of LOFAR, MeerKAT and other pathfinders, it is keenly felt. When I started my doctorate, the sheer scale of the task at hand felt overwhelming - to actually be able to contribute to its resolution seemed daunting indeed!

\pg
Thankfully, as the saying goes, no society sets for itself material goals which it cannot achieve. This thesis took place at an exciting time for radio interferometry: at the start of my doctorate, the LOFAR international stations were - to my knowledge - only beginning to be used, and even then, only tentatively; MeerKAT had not yet shown its first light; the techniques used throughout my work were still being developed. At the time of writing, great strides have been made. One of the greatest technical challenges of LOFAR - imaging using the international stations - is starting to become reality. This technical challenge is the key problem that this thesis set out to address. While we only achieved partial success so far, it is a testament to the difficulty of the task that it is not yet truly resolved.

\pg
One of the major results of this thesis is a model of a bright resolved source near a famous extragalactic field: properly modeling this source not only allows the use of international LOFAR stations, but also grants deeper access to the extragalactic field itself, which is otherwise polluted by the 3C source's sidelobes. This result was only achieved thanks to the other major result of this thesis: the development of a theoretical framework with which to better understand the effect of calibration errors on images made from interferometric data, and an algorithm to strongly mitigate them. 

%
%I owe a debt of gratitude to all my supervisors, whose efforts allowed me to grow (slowly and fitfully) into an adept of radio interferometry. I also owe a debt of gratitude to the European and South African radio-interferometric communities which took me into their fold, for it takes a village to raise a scientist - and I hope to one day be able to pay the favour forwards.

\pg
The structure of this manuscript is as follows: we begin with an introduction to radio interferometry, LOFAR, and the emission mechanisms which dominate for our field of interest. These introductions are primarily intended to give a brief overview of the technical aspects of the data reduced in this thesis. We follow with an overview of the Measurement Equation formalism, which underpins our theoretical work. This is the keystone of this thesis.

\pg
We then show the theoretical work that was developed as part of the research work done during the doctorate - which was published in Astronomy \& Astrophysics. Its practical application - a quality-based weighting scheme - is used throughout our data reduction. This data reduction is the next topic of this thesis: we contextualise the scientific interest of the data we reduce, and explain both the methods and the results we achieve. 

