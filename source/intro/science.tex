
\section{The Low-$\nu$ Sky: Emission Mechanisms}
\pg
This section aims to bring a brief introduction to the emission mechanisms which dominate at low frequencies, and thus determine the physics accessible to extragalactic astronomers working in this band. Specifically, it will briefly describe thermal radiation detected at low frequencies (associated with dust \& protoplanetary disks, and therefore a tracer of star formation) along with free-free radiation (which is emitted from ionised plasma outflows, and thus a tracer of various physical objects e.g. young stellar objects - cf. \citet{2017ApJ...834..206C} - and starburst regions - cf. \citet{2015A&A...574A.114V}) and synchrotron radiation, which is a tracer of more violent and energetic processes (and thus associated with supernova remnants, AGN, or radio halos - cf. \citet{2010A&A...509A..68C}).

\pg
Of course, each individual galaxy will have differing contributions from different mechanisms - in practice, when studying galactic populations, the overall flux at a given frequency will simply be summed up for each galaxy and become one point in a Spectral Energy Distribution, or SED. However, when studying individual objects, it can be critical to understand which emission mechanism dominates at what frequencies. %: it will be very difficult to find traces of star formation in the emission from a galaxy dominated by an AGN, for example. Along with a brief introduction of the main emission mechanisms, then, we will give the characteristic spectrum of each, to show how different types of emission can be differentiated in practice.
For example, the radio and far-infrared spectrum for nearby M82 are shown in Fig. \ref{plot.m82.spectrum}\footnote{Figure and work taken from the NRAO website. For further information, \href{https://www.cv.nrao.edu/course/astr534/FreeFreeEmission.html}{see here}}.
\begin{figure*}[!h]
\centering
\includegraphics[width=\textwidth]{images/M82Spectrum.png}
\caption{\label{plot.m82.spectrum} Radio and far-infrared spectrum for galaxy M82, as estimated \href{https://www.cv.nrao.edu/course/astr534/FreeFreeEmission.html}{by the NRAO online course}. The flat curve corresponds to free-free emission, while synchrotron radiation and thermal dust emission dominate at low and high frequencies respectively.}
\end{figure*}

\pg


\subsection{Thermal Radiation}
\pg
Also known as black-body radiation, its spectral intensity is given by Planck's law, given in Eq. \ref{eq.planck}.
\begin{equation}\label{eq.planck}
%B_\lambda (\lambda,T) = \frac{2hc^2}{\lambda^5}\left(e^{\frac{hc}{\lambda k_BT}}-1\right)^{-1}
B_\nu(\nu,T) = \frac{2h\nu^3}{c^2}\left(e^\frac{h\nu}{k_BT}-1\right)^{-1}
\end{equation}
where $B_\lambda$ is the flux density at frequency $\nu$ for a source with temperature $T$, and $k_B=1.381 \left[J/K\right]$ is the Boltzmann constant. Near protostellar disks, synchrotron emission is absorbed by the ambient interstellar medium, heating it up to an average of $\sim 10^4$ (see \citetads{2009ApJS..181..255A}, \citetads{1978ppim.book.....S} and references therein). This gives the following spectral curve:

\begin{figure*}[!h]
\centering
\includegraphics[width=0.8\textwidth]{images/ThermalEmission.png}
\caption{\label{plot.thermal}Log-Log plot of blackbody radiation emitted by a particle at $10^4$ Kelvin. Note that the y-axis is arbitrary, since it will in practice be modulated by the number of particles in a region, radiation efficiency, resolution etc.}
\end{figure*}
\pg
This is the dominant mode of emission for distant galaxies in the infrared band. In the LOFAR regime, it is not expected to dominate for extragalactic sources, but ought to be detected for resolved galaxies.

\subsection{Free-Free Radiation}
\pg
Free-free or ``bremsstrahlung" (German for ``braking") radiation occurs when the trajectory of a high-energy charged particle is deflected by an electric field. This non-thermal emission mechanism is the dominant mechanism in HII regions (which contain ionised hydrogen), where star formation has previously taken place. It is called free-free emission because it is produced by free electrons scattering off ionised hydrogen without being captured. % as shown in Fig. \ref{plot.freefree}
%\begin{figure*}[!h]
%\centering
%\includegraphics[width=0.3\textwidth]{images/bremsstrahlung.png}
%\caption{\label{plot.freefree} Schematic showing the physical process leading to free-free emission. As we can see, the electron is scattered, but not captured. Source: \url{https://thephysicsbehind.com/2015/04/16/x-ray-tubes/}}
%\end{figure*}

\pg
This emission's spectrum is heavily dependent on a number of factors: including frequency, temperature, and critically, free-free opacity $\tau_\nu$, itself a function of electron density. It is characterised by a knee in its spectrum, occurring where $\tau_\nu\sim 1$. This knee delineates two regions with different spectral indices; $\alpha \sim -0.1$ at higher frequencies, and $\alpha \leq 2$ at lower frequencies. This gives a characteristic shape, shown in Fig. \ref{plot.freefree.spectrum}.
\begin{figure*}[!h]
\centering
\includegraphics[width=0.9\textwidth]{images/freefree.png}
\caption{\label{plot.freefree.spectrum} Characteristic spectrum of free-free radiation. Arbitrary unit scale. Source: NRAO online course.}
\end{figure*}

\pg
As we can see, this spectrum falls off sharply with frequency. As such, while it is expected to dominate over thermal emission in the absence of synchrotron radiation, it is unlikely to dominate in the LOFAR band. It acts as a tracer for star formation and other "gentler" physical processes detected at low radio frequencies.

\subsection{Synchrotron Radiation}
\pg
Synchrotron radiation (or "magnetobremsstrahlung") occurs when the trajectory of a high-energy charged particle is deflected by a magnetic field. As the German name suggests, it is the magnetic equivalent of free-free radiation. It is the tracer of extremely violent processes, such as AGN jets. In its mildly relativistic regime, it is referred to as cyclotron radiation, after the device in which it was first tested, and in non-relativistic regimes, it is known as gyro radiation. 

\pg





