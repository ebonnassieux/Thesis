% !TeX spellcheck = en_UK

\section{Calibration Methods in Radio Interferometry}\label{section.calibration}
\pg
In this section, we will discuss the implementation of interferometric array calibration. Our analysis is based on the RIME formalism, described in \cref{section.RIME}. One key metric of calibration quality is the \emph{dynamic range} (DR). High dynamic ranges mean that a high contrast has been obtained, and fainter sources can be reached. Dynamic range is defined as the following:
\begin{equation}\label{eq.DR}
DR = \frac{flux_{max}}{\max(\sigma_{thermal},\sigma_{artefacts})}
\end{equation}
where $flux_{max}$ is the flux of the brightest source, $\sigma_{thermal}$ the thermal noise in the image, and $\sigma_{artefacts}$ the noise associated with calibration artefacts.

\pg
The distinction between these two noise sources is crucial; one can never go `below noise' for a given observation, no matter the quality of calibration. Astronomers typically observe for longer periods of time in order to reduce $\sigma_{thermal}$ in their images, but this will not reduce the artefacts caused by poor calibration solutions. However, uncorrected Direction-Dependent Effects will not go away on their own, no matter how long the integration time.

\pg
There are three `generations' of calibration methods, of increasing complexity. I shall describe them in terms of the RIME, showing how each generation increases in generality to account for more exotic effects. For the sake of clarity, I shall henceforth refer to them interchangeably as `nth-generation calibration' or `nGC' methods.

\subsection{Generational Analysis}

\subsubsection{First-Generation: Open-Loop Calibration}\label{section.calibration.1gc}

\pg
First-generation calibration methods (1GC methods) consist of open-loop calibration. This relies entirely on instrument stability, and thus imposes significant design constraints on radio telescopes. It consists of briefly observing a calibrator before and after each observation run to find a gain factor and offset error\footnote{For a concrete example, see \href{http://www.analog.com/en/analog-dialogue/articles/open-loop-calibration-techniques.html}{here}.}

\pg
Phase calibration in the 1GC era `proper' was not done, as engineers were capable of ensuring phase stability in contemporary interferometers. Phase was thus calculated relative to a fixed frame of reference, usually the central antenna of a 3-antenna array. 

\pg
In RIME terms, this consists of solving for a very basic form of $\Gjones_p$:
\begin{equation}
\Gjones_p = \begin{bmatrix} a_p t + b_p \end{bmatrix} % \begin{bmatrix} e^{2 \pi i \nu ( \phi_p - \phi_0)} \end{bmatrix}
\end{equation}
where $a_p,b_p$ are constants solved for during open-loop calibration and $t$ is time. %, $\nu$ the observing frequency, $\phi$ the phase at an antenna, and $\phi_0$ the phase at the reference antenna.

\pg
While values for $a_p$ and $b_p$ can in theory be found for both autocorrelation and both crosscorrelations, low signal-to-noise means that in practice, a single set of values is solved for per antenna\footnote{This reduces calibration to solving only for the intensity gains: the data can then only be used for \emph{intensity mapping} (e.g. \citepads{1957IAUS....4..159J}). This practice therefore precludes polarimetry.}. With these techniques, one can achieve dynamic ranges of about 100:1 (\citepads{2010A&A...524A..61N}).

\subsubsection{Second-Generation: Self-Calibration}\label{section.calibration.2gc}

\pg
Second-generation calibration methods (2GC methods) are defined by their use of \emph{self-calibration}\footnote{On the discovery of self-calibration and its evolution in parallel to adaptive optics, I highly recommend the chapter titled "The Almost Serendipitous Discovery of Self-Calibration'' in "\href{http://library.nrao.edu/public/collection/02000000000280.pdf}{\citep{serendipitous}}.}, commonly referred to as \emph{self-cal} (\citepads{1984ARA&A..22...97P}). As described in \cref{section.RIME.FullSky.CVZ}, this method can only be deployed if the brightness matrix of the sky is the same for all baselines\footnote{This is not to say that the brightness matrix must contain only point sources, but rather that moving our interferometer 500m to the East should not change its measured visibilities} (i.e. that we are not affected by direction-dependent effects).

\pg
The seeds of self-calibration (and adaptive optics) is a paper published in the era of 1GC by Jennison (\citepads{1958MNRAS.118..276J}, expanding on his PhD work, which was published in 1951). With sufficient signal-to-noise, he showed that \emph{phase closure} could be calculated and errors due to the atmosphere thus mitigated.

\pg
Self-calibration and adaptive optics are the same thing, albeit meeting different constraints. Most notably, interferometric data is digital, allowing radio astronomers to perform their adaptive optics correction after the observation.

\pg
[ TODO: put adaptive optics diagram here maybe]

\pg
If amplitude and phase gains can be written as antenna-dependent, then each antenna-based error is estimated N times (once per baseline which includes this antenna). By estimating, and correcting for, these errors, one can use a simple source model to infer an improved one. This is why the method is referred to as self-cal; by calibrating `on' a good calibrator source (bright, compact and unresolved), one can drastically improve one's source model, along with one's calibration solutions.

\pg
By extending this idea to VLBI\footnote{e.g.  \citepads{1974ApJ...193..293R}}, \emph{amplitude closure} was introduced to the field along with phase closure, as described in \citepads{1983Sci...219...51R}. Astronomers were quick to apply these methods to interferometers in general.

\pg
The great advantage of radio interferometry over optics, however, is that we can \emph{iterate} over progressively improved source models. This is because interferometric data is digital, and because we record phase information.

\pg
In practice, of course, self-cal will be limited by noise. More precisely, it will be limited by sensitivity, in the form of the signal-to-noise ratio (henceforth SNR). For sufficiently bright sources, with high SNR, self-calibration can easily improve dynamic range by a factor of 10 (\cite{serendipitous}, p. 154) in a single iteration. 

\subsubsection{Third-Generation: Direction-Dependent Effects}

\pg
Third-generation calibration (3GC) is an extension of 2GC calibration which takes direction-dependent effects into account. At the time of writing, this is the state of the art in radio interferometric calibration.

\pg
In this section, I will discuss the approach taken by [cite DDF and kMS paper when it comes out] and [cite prefactor paper if/when it is out], \emph{facet-based calibration}. The approach cannot be divorced from imaging, for the simple reason that solving for and applying Jones matrices in different directions requires image-plane knowledge of the sky brightness distribution $\Bmatrix$. The imaging challenges this methods introduces will be discussed in [href to relevant imaging section].

\pg







