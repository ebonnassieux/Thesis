% !TeX spellcheck = en_UK

\section{Calibrating and Imaging the Extended Groth Strip}\label{section.calibration}
\pg
In this section, we describe the work done as part of this PhD on calibrating and imaging interferometric data on the Boötes field, an extragalactic field with rich multi-wavelength coverage. It has been long observed as part of the All-Wavelength Extended Groth Strip International Survey collaboration (\citepads{2007ApJ...660L...1D}), which later became part of the CANDELS collaboration (\citepads{2011ApJS..197...35G}). The field is centred at $\alpha=14^h17^m,\delta=+52\deg 30'$, placing it between the tail of Ursa Major and Draco. It has notably been the subject of very deep Hubble Space Telescope observations, 

\newpage

\begin{table}[]
\centering
\caption{Multi-wavelength coverage of the Boötes field}
\label{bootes-coverage-table}
\begin{tabular}{|ccccc|} \hline
Catalog                    & $\lambda$,$\nu$,band           & Sensitivity              & Resolution               & Source                            \\\hline\hline
XBootes                    & 0.5-0.7 keV                    & $4(8).10^{-15}$ ergs cm$^{-2}\text{s}^{-1}$ & 0.492''                & \citepads{2005ApJS..161....1M}    \\\hline
\multirow{6}{*}{NOAO-Deep} & B$_\text{W}$                   & 26.6 mag                 & \multirow{6}{*}{1''\footnote{Seeing-limited. Value given at: \url{https://www.noao.edu/noao/noaodeep/DR3/optimagepropsdr3.html}}}                                              &\multirow{6}{*}{\citepads{1999ASPC..191..111J}}\\
                           & R                              & 25.8 mag                 &                          &                                   \\
                           & I                              & 25.5 mag                 &                          &                                   \\
                           & J                              & 20.2 mag                 &                          &                                   \\
                           & H                              & 19.6 mag                 &                          &                                   \\
                           & K                              & 19.5 mag                 &                          &                                   \\\hline
\multirow{4}{*}{WISE}      & $22\mu$m                       & 5.9 mJy                  & \multirow{4}{*}{1.375''\footnote{Value taken from the \hyperlink{http://wise2.ipac.caltech.edu/docs/release/allsky/expsup/sec1_2.html}{Executive Summary of WISE All-Sky Release Data Products}, under section I.2.a. Image Atlas}} &\multirow{4}{*}{\citepads{2012wise.rept....1C}}              \\
                           & $12\mu$m                       & 0.73 mJy                 &                          &                                   \\
                           & $4.6\mu$m                      & 0.1 mJy                  &                          &                                   \\
                           & $3.4\mu$m                      & 0.048 mJy                &                          &                                   \\\hline
WSRT                       & 1.4 GHz                        & 0.028 mJy                & 13''$\times$27''         & \citepads{2002AJ....123.1784D}    \\
VLA                        & 324.5 MHz                      & 0.5 mJy                  & 6''                      & \citepads{2015MNRAS.450.1477C}    \\
GMRT                       & 153 MHz                        & 5 mJy                    & 26''$\times$22''         & \citepads{2011AnA...535A..38I}    \\
LOFAR-HBA                  & 144 MHz                        & TBA                      & 1.5''                    &   TBA                             \\\hline
\multirow{3}{*}{LOFAR-LBA} & 62 MHz                         & 25 mJy                   & 4''                      & \multirow{3}{*}{\citepads{2014ApJ...793...82V}}    \\
                           & 46 MHz                         & 40 mJy                   & 5''                      &                                   \\
                           & 34 MHz                         & 60 mJy                   & 7''                      &                                   \\\hline
\end{tabular}
\end{table}

\newpage
 tada
\newpage
%
%talk about the field: location, size, history
%
%talk about multi-lambda coverage: put coverage figure here!
%
%\subsection{High-resolution imaging: calibrating the International Baselines}