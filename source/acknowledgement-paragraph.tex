\pg
A few paragraphs can never be room enough to express the debt of gratitude that I owe, let alone to the veritable crowd to whom I owe it. This section is thus by necessity incomplete - but I will nevertheless do my best to account, as much as possible, for those who made this thesis possible.

\pg
I'll begin with thanks to Cyril Tasse, Philippe Zarka and Oleg Smirnov. Though each will claim to have only done a small part of the work in guiding and teaching me over the course of this PhD, they have all contributed monumentally - it was not always easy, but it was certainly worthwhile. I would also like to thank all the members of my thesis defense, many of whom crossed borders to attend. I would particularly like to thank my examiners - Neal Jackson, Patrick Charlot and Ciara Ferrari - whose thorough comments and invaluable insights greatly improved the quality of this work.

\pg
Similarly, I owe an infinite debt to the post-doctoral fellows and colleagues I had the pleasure to work with and befriend. I would particularly like to thank Marcel Atemkeng, Julien Girard, Roger Deane, Kshitij Thorat, Trienko Grobler, Arun Aniyan and Alan Loh - both for putting up with my ridiculous imprecations to my computer terminal, and for helping me refine my understanding of my own work. Of course, I am just as grateful to have known and worked with many other fine people: Corentin Louis, Lucas Grosset, and Lisa Bardou in France; Theophilius Ansah-Narh, Kelachukwu Iheanetu, Benjamin Hugo, Ulrich Mbou Sob, Kwazi Mthembu and all the RATT Masters students during my time in South Africa.

\pg
I would also like to take the time to thank the people who made this thesis possible on a logistical level: nothing could have been done without Emmanuel Thetas and the nancep cluster. But on a more fundamental level, I would like to thank secretaries in both France and South Africa: without Jacqueline Plancy, Géraldine Gaillant, and Ronel Groenwald, neither the Observatoire de Paris nor the RATT would be what they are now.

\pg
Outside the lab, I would like to extend my warmest thanks to my wonderful partner Christine, for also putting up with my ridiculous imprecations to my computer at home and for holding on while I was in South Africa. I also thank my family, whose support was often invaluable - my parents Beatrice and Marc, and my sisters Violette and Coline. To my friends scattered across the world: Laith, Bob, Erith, Alex, Cameron, Craig, Ru, Alasdair, Rachel, Sammie, Michael, Jois, Poppy, Paul, Celine, Philippe, Helene, Sarah, Chloe, Prudence, Elise, Sacha, and many more besides; I am every day thankful to count myself your friend.

\pg
I am also extremely grateful to those teachers who first helped me develop an interest in science and mathematics: without Nasreddine Nouar, Katrina Kemper, and Judith Byrd, I would not be what I am today. I hope to be as good a teacher to my own students as you were to me. Similarly, I would like to thank those who encouraged me to pursue this interest in higher education: Catherine Heymans, Andreas Zech and Jacques le Bourlot. Yours is an example I strive to emulate.

\pg
Finally, to close this manuscript on a light note, I want to extend my heartfelt gratitude to the cats who have graced my life, at home and abroad. May they purr blissfully for many years to come.