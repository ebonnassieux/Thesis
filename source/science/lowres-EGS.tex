\chapter{The Groth Strip with Dutch LOFAR}\label{section.EGS.lowres}
\minitoc
% % % % % % % % % % % % % % % % % % % % % % %
\section{Aims \& Methodology}

\pg
In this section, we describe the work done to make a wide-field image of the entire LOFAR primary beam, centred on EGS. Such a large-field, low-frequency image of this field has never yet been done in this frequency band. Due to technical limitations, the maximum attainable image size (in terms of number of pixels) will limit resolution to about $5''$. As such, the international stations are not used for this part of the project.

\pg
A more complete sky model can be created from this image, allowing for one last round of self-calibration on the 3C295 high-resolution model. Bright out-of-field sources can be subtracted from images of the Extended Groth Strip so that their sidelobes do not contaminate the final images. To achieve this, we require direction-dependent calibration. Because we do not use the international LOFAR stations, we do not use our high-resolution model of 3C295. This ensures that we do not ``hard-set" errors in our high-resolution model caused by other sources in the field. 

\pg
Our methodology is therefore to calibrate the full LOFAR bandwidth using the LOFAR Surveys KSP pipeline (talk about DD calibration, facetting, etc). We then perform source analysis on the most interesting objects in the field, using ancillary data from NVSS \citepads[NRAO VLA Sky Survey, ]{1998AJ....115.1693C}, SDSS \citepads[Sloan Digital Sky Survey, ]{2000AJ....120.1579Y} and WISE \citepads{2010AJ....140.1868W} surveys and cross-referenced using SIMBAD \citepads{2000A&AS..143....9W}. Overlays for those sources will be shown in \cref{sec.lowresEGS.overlays}.


% % % % % % % % % % % % % % % % % % % % % % %
\section{Data Reduction}

\subsection{Data \& Observation Properties}
\pg
The data used to make this image are the same as used to make the high-resolution model of 3C295 in \cref{section.3c295}, but we do not use the visibilities of baselines including international stations in our calibration or imaging. They were acquired through the LOFAR Long-Term Archive service. We do not phase-shift our data when imaging.

\subsection{Calibration Strategy}

\pg
An initial sky model is generated using the VLSSr \citepads[VLA Low-frequency Sky Survey - redux, ]{2012RaSc...47.0K04L}, WENSS \citepads[WEsterbork Northern Sky Survey]{1997A&AS..124..259R} and the NVSS \citepads[NRAO VLA Sky Survey, ]{1998AJ....115.1693C}. This model serves as the basis for facet-based self-calibration, performed using the killMS-DDF surveys pipeline \citepads{2017A&A...598A.104S}.
\pg
The direction-dependent calibration was done using \textcolor{red}{[N]} facets, spread as shown in \textcolor{red}{fig. make figure}. Some facets converge faster or better than others. Calibration is done solving for full, complex Jones matrices, rather than a physics-based or diagonal-and-rotation matrix [CHECK THAT THIS IS TRUE!]. Imaging is done [GIVE WEIGHTING, ETC].

\pg
\textcolor{red}{show self-cal loop (before-after) in images \& gains; show improvement from direction-dependent cal. on entire field \& individual sources}



% % % % % % % % % % % % % % % % % % % % % % %
\section{Overlays \& Images}\label{sec.lowresEGS.overlays}

\subsection{Selected Sources}

\pg
We select 12 sources in the wide-field image. Our selection is primarily based on whether sources have interesting or peculiar diffuse emission, though some are chosen for being particularly classic examples of radio galaxies. We scan the image from south to north and east to west (scanning upwards then shifting to the right in the image). The source positions are given in \cref{table.egs.sources}.

\begin{table}[h!]
\begin{tabular}{ccccc}
Name    & RA [hms]    & Dec [dms]   & Likely Association            & LOFAR thumbnail \\\hline
EGS-1   & 14:37:39.53 & 53:36:31.24 & NVSS Radio Galaxy             & \cref{fig.egs1.lofarim} \\
EGS-2   & 14:35:27.84 & 55:07:56.32 & Radio Galaxy in Cluster       & \cref{fig.egs2.lofarim} \\ 
EGS-3   & 14:29:34.13 & 54:43:46.93 & Radio Galaxy in Cluster       & \cref{fig.egs3.lofarim} \\
EGS-4   & 14:31:36.62 & 52:27:33.75 &                          & \cref{fig.egs4.lofarim} \\
EGS-5   & 14:29:48.89 & 51:10:30.52 &                          & \cref{fig.egs5.lofarim} \\
EGS-6   & 14:26:04.09 & 51:29:35.07 &                          & \cref{fig.egs6.lofarim} \\
EGS-7   & 14:17:55.58 & 50:08:01.75 &                          & \cref{fig.egs7.lofarim} \\
EGS-8   & 14:14:40.42 & 51:17:41.00 &                          & \cref{fig.egs8.lofarim} \\
EGS-9   & 14:11:36.43 & 52:54:25.53 &                          & \cref{fig.egs9.lofarim} \\
EGS-10  & 14:07:09.90 & 55:04:22.23 &                          & \cref{fig.egs10.lofarim} \\
EGS-11  & 14:03:16.00 & 51:43:35.23 &                          & \cref{fig.egs11.lofarim} \\
EGS-12  & 14:02:43.49 & 51:03:14.69 &                          & \cref{fig.egs12.lofarim} \\
\end{tabular}
\caption{\label{table.egs.sources} Table recapitulating the names, positions and likely associations of all 12 chosen sources in the primary beam. These sources were primarily chosen because they had peculiar or interesting diffuse emission. The associated LOFAR image thumbnails are also given for reference.}
\end{table}

\subsection{Images \& Cross-matching}

\clearpage
\subsubsection{EGS-1}


\begin{figure}[h!]
\centering
\begin{subfigure}{.48\textwidth}
\resizebox{\hsize}{!}{\includegraphics{images/{egs1.LOFARegs}.png}}
\caption{\label{fig.egs1.lofarim} Cutout of our LOFAR wide-field image centred on EGS-1.}
\end{subfigure}
\hfill
\begin{subfigure}{.48\textwidth}
\resizebox{\hsize}{!}{\includegraphics{images/{egs1.mosaic.fits.lofar}.png}}
\caption{\label{fig.egs1.sdss.lofoverlay} SDSS image of EGS-1, with \cref{fig.egs1.lofarim} as an overlay.}
\end{subfigure}
\hfill
\begin{subfigure}{.48\textwidth}
\resizebox{\hsize}{!}{\includegraphics{images/{egs1.mosaic.fits.nvss}.png}}
\caption{\label{fig.egs1.sdss.nvssoverlay} Same image as \cref{fig.egs1.sdss.lofoverlay}, with NVSS overlay.}
\end{subfigure}
\hfill
\begin{subfigure}{.48\textwidth}
\resizebox{\hsize}{!}{\includegraphics{images/{egs1.mosaic.fits.Wise3.4}.png}}
\caption{\label{fig.egs1.sdss.wiseoverlay} Same image as \cref{fig.egs1.sdss.lofoverlay}, with WISE overlay.}
\end{subfigure}
\caption{\label{fig.egs1} Images \& Overlays for EGS-1.}
\end{figure}

\pg
EGS-1 was selected from the wide-field LOFAR image because it showed interesting diffuse structure. \cref{fig.egs1} shows four images: a cutout of the LOFAR image around the source, an SDSS cutout of the same area with the LOFAR image superimposed as an overlay, the same SDSS cutout with an NVSS image overlayed, and finally the same SDSS image again with a WISE image overlay. The first image aims to show why the source was selected. The second shows optical counterparts to LOFAR emission. The third serves to check whether we pick up similar structure and emission in NVSS images as the LOFAR wide-field image. Note that the NVSS images are at a much lower resolution, and are therefore not expected to give any information on LOFAR source structure, but can be a useful way to ensure that we are not choosing artefacts or spurious emission as sources to investigate. Finally, the WISE overlay attempts to see whether infrared emission can be associated to these sources. Note that the sensitivity of SDSS images are not homogeneous: this is because not all patches of the sky have received the same coverage. 

\pg
EGS-1 does not seem to match up to any optical source, nor does it match an infrared source. EGS-1 therefore seems to be best identified as a low-frequency counterpart to NVSS J143706+533225. Its shape seems to include two jets, one much brighter than the other, but both visible in \cref{fig.egs1.lofarim}. This is congruent with a radio galaxy with one Doppler-boosted jet. The absence of an optical counterpart could be due to dust extinction or to insufficient sensitivity. 

%\clearpage
\subsubsection{EGS-2}

\begin{figure}[h!]
\centering
\begin{subfigure}{.48\textwidth}
\resizebox{\hsize}{!}{\includegraphics{images/{egs2.LOFARegs}.png}}
\caption{\label{fig.egs2.lofarim} Cutout of our LOFAR wide-field image centred on EGS-2.}
\end{subfigure}
\hfill
\begin{subfigure}{.48\textwidth}
\resizebox{\hsize}{!}{\includegraphics{images/{egs2.mosaic.fits.lofar}.png}}
\caption{\label{fig.egs2.sdss.lofoverlay} SDSS image of EGS-2, with \cref{fig.egs2.lofarim} as an overlay.}
\end{subfigure}
\hfill
\begin{subfigure}{.48\textwidth}
\resizebox{\hsize}{!}{\includegraphics{images/{egs2.mosaic.fits.nvss}.png}}
\caption{\label{fig.egs2.sdss.nvssoverlay} Same image as \cref{fig.egs2.sdss.lofoverlay}, with NVSS overlay.}
\end{subfigure}
\hfill
\begin{subfigure}{.48\textwidth}
\resizebox{\hsize}{!}{\includegraphics{images/{egs2.mosaic.fits.Wise3.4}.png}}
\caption{\label{fig.egs2.sdss.wiseoverlay} Same image as \cref{fig.egs2.sdss.lofoverlay}, with WISE overlay.}
\end{subfigure}
\caption{\label{fig.egs2} Images \& Overlays for EGS-2.}
\end{figure}


\pg
EGS-2 was selected from the wide-field LOFAR image because of the interestingly turbulent morphology of its northern jet, and extended diffuse emission. \cref{fig.egs2} shows four images: a cutout of the LOFAR image around the source, an SDSS cutout of the same area with the LOFAR image superimposed as an overlay, the same SDSS cutout with an NVSS image overlayed, and finally the same SDSS image again with a WISE image overlay. The first image aims to show why the source was selected. The second shows optical counterparts to LOFAR emission. The third serves to check whether we pick up similar structure and emission in NVSS images as the LOFAR wide-field image. Note that the NVSS images are at a much lower resolution, and are therefore not expected to give any information on LOFAR source structure, but can be a useful way to ensure that we are not choosing artefacts or spurious emission as sources to investigate. Finally, the WISE overlay attempts to see whether infrared emission can be associated to these sources. Note that the sensitivity of SDSS images are not homogeneous: this is because not all patches of the sky have received the same coverage. 


\pg
A bright optical source lies very close to the mid-point of EGS-2. This source - 2MASX J14352846+5507519\footnote{\url{http://simbad.u-strasbg.fr/simbad/sim-id?Ident=\%40499314&Name=2MASX\%20J14352846\%2b5507519&submit=submit}} - is the brightest galaxy in a cluster. It lies at a redshift of 0.13985 \citepads{2013yCat.5139....0A}. Furthermore, there is an NVSS source which seems to nicely overlap the EGS-2: NVSS J143527+550756 \citepads{1998AJ....115.1693C}. These two sources do not seem to be associated in the literature. [reasons why they aren't?] 

\pg
The northern edge of EGS-2's upper jet has interesting morphology: if EGS-2 is indeed associated to the brightest galaxy in the cluster, then this morphology is likely due to differing densities of the intercluster and intracluster media, which the jets will penetrate and interact with in different ways. This morphology is lost on the overlay, and so may not be statistically significant enough to draw conclusions from, however. As ever, because this source lies far from the observation pointing, the sensitivity and calibration are not optimal for analysis: further pointings on EGS-2 might be required to achieve the signal-to-noise required to extract useful information, especially using international LOFAR.

%\clearpage
\subsubsection{EGS-3}

\pg
EGS-3 is one of the most impressive sources in the primary beam.It has very complex diffuse structure. It almost certainly consists of two jets, though the interactions of these jets with the neighbouring medium must be complex indeed. From \cref{fig.egs3.sdss.nvssoverlay}, one can see that the emission can be successfully matched to no less than two individual NVSS sources. In other words, if those NVSS sources are linked to the LOFAR source - and given their spatial distribution, they almost certainly are - then they are two components of a single, deeper structure. The infrared image tells us very little.


\begin{figure}[h!]
\centering
\begin{subfigure}{.48\textwidth}
\resizebox{\hsize}{!}{\includegraphics{images/{egs3.LOFARegs}.png}}
\caption{\label{fig.egs3.lofarim} Cutout of our LOFAR wide-field image centred on EGS-3.}
\end{subfigure}
\hfill
\begin{subfigure}{.48\textwidth}
\resizebox{\hsize}{!}{\includegraphics{images/{egs3.mosaic.fits.lofar}.png}}
\caption{\label{fig.egs3.sdss.lofoverlay} SDSS image of EGS-3, with \cref{fig.egs3.lofarim} as an overlay.}
\end{subfigure}
\hfill
\begin{subfigure}{.48\textwidth}
\resizebox{\hsize}{!}{\includegraphics{images/{egs3.mosaic.fits.nvss}.png}}
\caption{\label{fig.egs3.sdss.nvssoverlay} Same image as \cref{fig.egs3.sdss.lofoverlay}, with NVSS overlay.}
\end{subfigure}
\hfill
\begin{subfigure}{.48\textwidth}
\resizebox{\hsize}{!}{\includegraphics{images/{egs3.mosaic.fits.Wise3.4}.png}}
\caption{\label{fig.egs3.sdss.wiseoverlay} Same image as \cref{fig.egs3.sdss.lofoverlay}, with WISE overlay.}
\end{subfigure}
\caption{\label{fig.egs3} Images \& Overlays for EGS-3.}
\end{figure}


\pg
Note that an optical source lies along the line linking the two diffuse poles of EGS-3: this source is SDSS J142933.44+544335.2, the brightest galaxy in a cluster. If EGS-3 is a radio galaxy with jet emission, which it almost certainly is (the rough symmetry of the two ``cotton balls" on either side of a ``rod"-like component strongly indicate that this is a very warped jet), then its optical emission must lie somewhere within the ``rod". That this emission would be associated with violent interactions between jet-accelerated particles and ambiant inter-cluster and intra-cluster media seems likely: it would explain the complex, turbulent distribution of emission and the strong ``warping" of the jet structure into the whirls and eddies seen in \cref{fig.egs3.lofarim}. EGS-3 can therefore be fairly unambiguously associated to a galaxy cluster.

\clearpage
\subsubsection{EGS-4}


\begin{figure}[h!]
\centering
\begin{subfigure}{.48\textwidth}
\resizebox{\hsize}{!}{\includegraphics{images/{egs4.LOFARegs}.png}}
\caption{\label{fig.egs4.lofarim} Cutout of our LOFAR wide-field image centred on EGS-4.}
\end{subfigure}
\hfill
\begin{subfigure}{.48\textwidth}
\resizebox{\hsize}{!}{\includegraphics{images/{egs4.mosaic.fits.lofar}.png}}
\caption{\label{fig.egs4.sdss.lofoverlay} SDSS image of EGS-4, with \cref{fig.egs4.lofarim} as an overlay.}
\end{subfigure}
\hfill
\begin{subfigure}{.48\textwidth}
\resizebox{\hsize}{!}{\includegraphics{images/{egs4.mosaic.fits.nvss}.png}}
\caption{\label{fig.egs4.sdss.nvssoverlay} Same image as \cref{fig.egs4.sdss.lofoverlay}, with NVSS overlay.}
\end{subfigure}
\hfill
\begin{subfigure}{.48\textwidth}
\resizebox{\hsize}{!}{\includegraphics{images/{egs4.mosaic.fits.Wise3.4}.png}}
\caption{\label{fig.egs4.sdss.wiseoverlay} Same image as \cref{fig.egs4.sdss.lofoverlay}, with WISE overlay.}
\end{subfigure}
\caption{\label{fig.egs4} Images \& Overlays for EGS-4.}
\end{figure}

\pg
Another impressive source, EGS-4 was selected for its complex diffuse structure. Like EGS-3, the structure of this source seems to indicate that it would be a radio jet galaxy within a galaxy cluster. Its LOFAR structure matches the lower-resolution NVSS structure, which is a strong indication that it is not spurious. By the same argument as for EGS-3, we expect this diffuse emission to have associated optical emission somewhere along its central axis. As we can see from \cref{fig.egs4.sdss.nvssoverlay}, EGS-4 can be associated to an NVSS source: NVSS J143137+522728. This radio galaxy lies at redshift 0.292 \citepads{2010ApJ...723.1119L}, and is itself associated to the NVSS source in the middle of EGS-4 and its WISE counterpart. SIMBAD has no further association for this radio galaxy.
Further pointings will be required to acquire more information on this source and its environment.

\clearpage
\subsubsection{EGS-5}

\begin{figure}[h!]
\centering
\begin{subfigure}{.48\textwidth}
\resizebox{\hsize}{!}{\includegraphics{images/{egs5.LOFARegs}.png}}
\caption{\label{fig.egs5.lofarim} Cutout of our LOFAR wide-field image centred on EGS-5.}
\end{subfigure}
\hfill
\begin{subfigure}{.48\textwidth}
\resizebox{\hsize}{!}{\includegraphics{images/{egs5.mosaic.fits.lofar}.png}}
\caption{\label{fig.egs5.sdss.lofoverlay} SDSS image of EGS-5, with \cref{fig.egs5.lofarim} as an overlay.}
\end{subfigure}
\hfill
\begin{subfigure}{.48\textwidth}
\resizebox{\hsize}{!}{\includegraphics{images/{egs5.mosaic.fits.nvss}.png}}
\caption{\label{fig.egs5.sdss.nvssoverlay} Same image as \cref{fig.egs5.sdss.lofoverlay}, with NVSS overlay.}
\end{subfigure}
\hfill
\begin{subfigure}{.48\textwidth}
\resizebox{\hsize}{!}{\includegraphics{images/{egs5.mosaic.fits.Wise3.4}.png}}
\caption{\label{fig.egs5.sdss.wiseoverlay} Same image as \cref{fig.egs5.sdss.lofoverlay}, with WISE overlay.}
\end{subfigure}
\caption{\label{fig.egs5} Images \& Overlays for EGS-5.}
\end{figure}

\pg
EGS-5 was chosen due to its size, and was expected to be a very classic radio galaxy. In fact, only its southern lobe was matched successfully to any other candidate using SIMBAD - NVSS J142941+510850. The northern lobe also clearly shows up in NVSS overlays. It is unfortunate that the centre of EGS-5 lies in a region of poor SDSS coverage. As it stands, the southern lobe can successfully be matched to J142941+510850, and the northern lobe to no source in SIMBAD. 


\clearpage
\subsubsection{EGS-6}


\begin{figure}[h!]
\centering
\begin{subfigure}{.48\textwidth}
\resizebox{\hsize}{!}{\includegraphics{images/{egs6.LOFARegs}.png}}
\caption{\label{fig.egs6.lofarim} Cutout of our LOFAR wide-field image centred on EGS-6.}
\end{subfigure}
\hfill
\begin{subfigure}{.48\textwidth}
\resizebox{\hsize}{!}{\includegraphics{images/{egs6.mosaic.fits.lofar}.png}}
\caption{\label{fig.egs6.sdss.lofoverlay} SDSS image of EGS-6, with \cref{fig.egs6.lofarim} as an overlay.}
\end{subfigure}
\hfill
\begin{subfigure}{.48\textwidth}
\resizebox{\hsize}{!}{\includegraphics{images/{egs6.mosaic.fits.nvss}.png}}
\caption{\label{fig.egs6.sdss.nvssoverlay} Same image as \cref{fig.egs6.sdss.lofoverlay}, with NVSS overlay.}
\end{subfigure}
\hfill
\begin{subfigure}{.48\textwidth}
\resizebox{\hsize}{!}{\includegraphics{images/{egs6.mosaic.fits.Wise3.4}.png}}
\caption{\label{fig.egs6.sdss.wiseoverlay} Same image as \cref{fig.egs6.sdss.lofoverlay}, with WISE overlay.}
\end{subfigure}
\caption{\label{fig.egs6} Images \& Overlays for EGS-6.}
\end{figure}

\pg
EGS-6 was also chosen as it seemed like a very standard, classical radio-strong galaxy. Its two big lobes have matches in NVSS, but there is no likely counterpart in the SDSS image, nor in the WISE overlay. The same holds for the smaller radio galaxy to the north-west.


\clearpage
\subsubsection{EGS-7}

\begin{figure}[h!]
\centering
\begin{subfigure}{.48\textwidth}
\resizebox{\hsize}{!}{\includegraphics{images/{egs7.LOFARegs}.png}}
\caption{\label{fig.egs7.lofarim} Cutout of our LOFAR wide-field image centred on EGS-7.}
\end{subfigure}
\hfill
\begin{subfigure}{.48\textwidth}
\resizebox{\hsize}{!}{\includegraphics{images/{egs7.mosaic.fits.lofar}.png}}
\caption{\label{fig.egs7.sdss.lofoverlay} SDSS image of EGS-7, with \cref{fig.egs7.lofarim} as an overlay.}
\end{subfigure}
\hfill
\begin{subfigure}{.48\textwidth}
\resizebox{\hsize}{!}{\includegraphics{images/{egs7.mosaic.fits.nvss}.png}}
\caption{\label{fig.egs7.sdss.nvssoverlay} Same image as \cref{fig.egs7.sdss.lofoverlay}, with NVSS overlay.}
\end{subfigure}
\hfill
\begin{subfigure}{.48\textwidth}
\resizebox{\hsize}{!}{\includegraphics{images/{egs7.mosaic.fits.Wise3.4}.png}}
\caption{\label{fig.egs7.sdss.wiseoverlay} Same image as \cref{fig.egs7.sdss.lofoverlay}, with WISE overlay.}
\end{subfigure}
\caption{\label{fig.egs7} Images \& Overlays for EGS-7.}
\end{figure}

\pg
This source was chosen because of its very strong lobes and classic shape. It is clearly associated with 2MASX J14175515+5008041, which lies at $z=.186414$. This source was picked up in LBA \citepads{2014ApJ...793...82V}. Note the resolution improvement over NVSS, even without international LOFAR.

\clearpage
\subsubsection{EGS-8}

\begin{figure}[h!]
\centering
\begin{subfigure}{.48\textwidth}
\resizebox{\hsize}{!}{\includegraphics{images/{egs8.LOFARegs}.png}}
\caption{\label{fig.egs8.lofarim} Cutout of our LOFAR wide-field image centred on EGS-8.}
\end{subfigure}
\hfill
\begin{subfigure}{.48\textwidth}
\resizebox{\hsize}{!}{\includegraphics{images/{egs8.mosaic.fits.lofar}.png}}
\caption{\label{fig.egs8.sdss.lofoverlay} SDSS image of EGS-8, with \cref{fig.egs8.lofarim} as an overlay.}
\end{subfigure}
\hfill
\begin{subfigure}{.48\textwidth}
\resizebox{\hsize}{!}{\includegraphics{images/{egs8.mosaic.fits.nvss}.png}}
\caption{\label{fig.egs8.sdss.nvssoverlay} Same image as \cref{fig.egs8.sdss.lofoverlay}, with NVSS overlay.}
\end{subfigure}
\hfill
\begin{subfigure}{.48\textwidth}
\resizebox{\hsize}{!}{\includegraphics{images/{egs8.mosaic.fits.Wise3.4}.png}}
\caption{\label{fig.egs8.sdss.wiseoverlay} Same image as \cref{fig.egs8.sdss.lofoverlay}, with WISE overlay.}
\end{subfigure}
\caption{\label{fig.egs8} Images \& Overlays for EGS-8.}
\end{figure}

\pg
EGS-8 was chosen because it is a complex diffuse source. Note that the source to its South is contaminated by a lot of artefacts in the LOFAR image - but that there definitely is a source there, as there is a NVSS counterpart. There is no WISE counterpart to EGS-8. It has two likely candidates for counterparts: first, J141440.4+511743 \citepads{2011A&A...530A..60M}, a radio galaxy lying a few arcseconds to the North of the centre of EGS-8. Second, SDSSCGB 5990 \citepads{2009MNRAS.395..255M}, a compact group of galaxies, which lies a few arcseconds to the South-East. The redshift of both objects are unknown, so it is unclear whether EGS-8 could represent the interaction between both sources, but J141440.4+511743 is likely the host galaxy for EGS-8's radio lobes.
\clearpage
\subsubsection{EGS-9}

\begin{figure}[h!]
\centering
\begin{subfigure}{.48\textwidth}
\resizebox{\hsize}{!}{\includegraphics{images/{egs9.LOFARegs}.png}}
\caption{\label{fig.egs9.lofarim} Cutout of our LOFAR wide-field image centred on EGS-9.}
\end{subfigure}
\hfill
\begin{subfigure}{.48\textwidth}
\resizebox{\hsize}{!}{\includegraphics{images/{egs9.mosaic.fits.lofar}.png}}
\caption{\label{fig.egs9.sdss.lofoverlay} SDSS image of EGS-9, with \cref{fig.egs9.lofarim} as an overlay.}
\end{subfigure}
\hfill
\begin{subfigure}{.48\textwidth}
\resizebox{\hsize}{!}{\includegraphics{images/{egs9.mosaic.fits.nvss}.png}}
\caption{\label{fig.egs9.sdss.nvssoverlay} Same image as \cref{fig.egs9.sdss.lofoverlay}, with NVSS overlay.}
\end{subfigure}
\hfill
\begin{subfigure}{.48\textwidth}
\resizebox{\hsize}{!}{\includegraphics{images/{egs9.mosaic.fits.Wise3.4}.png}}
\caption{\label{fig.egs9.sdss.wiseoverlay} Same image as \cref{fig.egs9.sdss.lofoverlay}, with WISE overlay.}
\end{subfigure}
\caption{\label{fig.egs9} Images \& Overlays for EGS-9.}
\end{figure}

\pg
This source was chosen because 

\subsubsection{EGS-10}

\subsubsection{EGS-11}

\subsubsection{EGS-12}







\clearpage