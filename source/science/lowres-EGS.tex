\chapter{Imaging the Extended Groth Strip without International Stations}

% % % % % % % % % % % % % % % % % % % % % % %
\section{Aims \& Methodology}

\pg
Here, we want to make wide-field image of entire LOFAR primary beam, centred on EGS, *at once*. Such image never been done in this freq. band
\pg
This requires direction-dependent calibration, which in turns requires good direction-independent calibration starting from good model of 3C295. This is because this source is brightest in primary beam, and its sidelobes will dominate in the image unless properly subtracted
\pg
Source extraction using model made from this low-res image will allow us to subtract neighbouring sources when imaging the EGS at high res.
\pg
Technical limitations in computing means that maximum attainable image size (in number of pixels) will limit pixel resolution to ~5'' (actual number to be determined properly!) - this in turn means that the international stations are not useful for imaging, as the resolution they allow will not be reachable due to these technical, computational limits.
\pg
methodology: calibrate full LOFAR bandwidth using high-res model from section above, done using the LOFAR Surveys KSP pipeline (talk about DD calibration, facetting, etc)
finally, overlays of sources in the field compared with VLA, optical, etc


% % % % % % % % % % % % % % % % % % % % % % %
\section{Data Reduction}

\subsection{Data \& Observation Properties}
same as before, same dataset

\subsection{Calibration Strategy}

\pg
start w/ D.I. calibration using model from before.
\pg
follow with D.D. calibration w/ lofar pipeline - describe the pipeline and its properties
\pg
show self-cal loop (before-after) in images \& gains; show improvement from direction-dependent cal. on entire field \& individual sources



% % % % % % % % % % % % % % % % % % % % % % %
\section{Overlays \& Images}

\pg
show overlays of various radio sources in the field over their optical, IR, X-ray, etc counterparts. Discuss astrometric accuracy, calibration quality, morphology, etc etc


\newpage