\chapter{Scientific Overview: Goals \& Strategy}
\minitoc
%\pg
%This section aims to provide a general overview of our scientific aims \& chosen strategy. [describe more]

\section{Science Goal}
\pg
So far, we have shown the technical work done on interferometric techniques as part of this PhD. In this section, we outline the application of this work, along with other modern tools, to the creation of large, deep and high-resolution surveys of the sky. While this approach is potentially extremely rewarding in terms of the science one can do with its resulting maps, it is also uniquely complicated.

\pg
Why, then, use this approach? One reason is, in fact, for the challenge itself. A high-resolution (matching HST resolution), wide-area (multiple degrees wide), high-quality radio survey has \emph{never} been achieved at the time of writing. Many surveys of the radio sky exist, of course, but they tend to combine only two of those three qualities. FIRST\citepads[see][and references therein]{2015ApJ...801...26H}, for example, has achieved tremendous results with a $4''$ resolution, relatively low sensitivity, and no short baselines (thereby missing diffuse emission and extended structure). The VLASS \citepads[see][and references therein]{2013arXiv1312.4602H} will aim to go to a $2.5''$ resolution, medium sensitivity, and few short baselines. %Finally, there will of course be the SKA1-MID survey once the SKA is operational; this survey is expected to provide excellent resolution, sensitivity, and good $uv$-coverage, making it sensitive to diffuse emission as well as point sources.

\pg
In this context, a sky survey made using international LOFAR would be competitive until the SKA-MID survey (and considering it maps the Northern sky rather than the Southern sky, would arguably remain competitive even then). It would also allow LOFAR to fulfill the role of pathfinder in a very literal way, by providing datasets on which algorithms and imaging strategies could be tested before handling SKA data volumes.

\pg
Making a high-resolution full-sky survey using international LOFAR is thus arguably useful in and of itself in terms of interferometric techniques \& instrumentation. What then of its science value? The two most obvious advantages of a good large-sky survey are that they are complete (i.e. they are an accurate sample of the underlying source distribution) while still giving information on rare outlier objects which may happen to lie in the field. This gives scientists access to a wealth of statistics from which to derive more robust estimations of population distributions and associated scientific measures of interest. Furthermore, depending on the quality of a given survey, it might even be able to provide such statistics simultaneously for various different fields of scientific interest.

\pg
Of course, LOFAR is already providing such catalogues - LOTSS \citepads[see][and references therein]{2017A&A...598A.104S}  is only the first of many to come. But these catalogues do not make use of the LOFAR international stations, and thus do not make full use of LOFAR's resolution ($5''$ resolution vs. $0.5''$).  This is so due to the technical challenges introduced by the use of international stations. What would they add to a LOFAR catalogue?

\pg
A high-resolution sky survey has all the advantages listed above with more besides. For AGN science, for example, the higher resolution gives information on smaller scales, which is very relevant to study the turbulent processes in the lobes and the nearby intergalactic medium. We also know that there exist small AGNs: size therefore matters, and resolving these small AGNs (which tend to lie at larger redshifts, hence their smaller angular size) can give extremely salient information on AGN population distributions as a function of the age of the universe. Indeed, for AGN science, the LOFAR international baselines can be the difference between having absolutely no information on size distribution (no resolved AGN) and knowing everything about the AGN size distribution within a given extragalactic field. Finally, resolving sources is very relevant to spectral studies of AGN populations: we know very little about the physics of AGNs at low frequencies. Information on large populations of resolved sources (allowing us to separate compact structure from extended emission) is a prerequisite to begin the statistical work needed to understand these low-frequency AGN physics. As such, the international baselines can bridge a critical gap in the data available to scientists studying AGN science.

\pg
AGN science is far from the only field which could benefit from the sub-arcsecond resolution that international LOFAR could provide. High-resolution imaging can be of tremendous interest to the study of star-forming galaxies: it could give insight into cosmic ray structure and access to spectral information. This is relevant because the entire star-forming history of galaxies is encoded in radio emission, but can only be accessed with sufficient spectral information and spatial resolution. A high-resolution survey, in particular, allows comparisons between optical and radio emission for known star-forming galaxies on a truly industrial scale, which would allow for the mapping of free-free absorption, supernova remnants, HII regions, etc...onto optical maps, and this for sources going up to high redshifts.

\pg
Of course, this does not run the full range of science cases which would benefit from high-resolution maps of large parts of the radio sky. Scientists studying gravitational lensing would benefit greatly from large samples of lensed galaxies. Resolved extragalactic recombination lines (for bright objects) would doubtlessly interest scientists studying galactic evolution. We expect these samples to be a natural byproduct of a large, high-resolution map of the radio sky: if they do not appear, that could indeed be a very interesting scientific result in and of itself.

\pg
More generally, high-resolution surveys provide high-resolution images of everyone's favourite objects and sources. It allows for better optical matching and identification, which is particularly relevant e.g. when needing to associate either a low-z or high-z sources to optical counterpart sources (in the case of the Extended Groth Strip, for example, using the international stations means matching the Hubble Space Telescope's resolution). They therefore provide a much more useful contribution to multi-wavelength datasets. It also improves the image's sensitivity to compact sources embedded in diffuse emission: if the source is better-resolved, then the emission associated with compact sources is more easily separated from emission emitted by its surroundings. This can be of tremendous help in image interpretation. Finally, actually resolving objects allows scientists to have better morphological classification for compact objects, which can be extremely useful e.g. when interested in AGN populations vs star-forming galaxy populations.




\section{3C295 and the Extended Groth Strip}

\pg
In this section, we briefly describe previous observations of the Extended Groth Strip, an extragalactic field with a rich multi-wavelength coverage described in Table \ref{table.EGS.observation}. It has been long observed as part of the All-Wavelength Extended Groth Strip International Survey collaboration (\citepads{2007ApJ...660L...1D}), which later became part of the CANDELS collaboration (\citepads{2011ApJS..197...35G}). The field is centred at $\alpha=14^h17^m,\delta=+52\deg 30'$, placing it between the tail of Ursa Major and Draco. Its size is $0.7'\times 0.1'$. It has notably been the subject of very deep Hubble Space Telescope observations, which the LOFAR VLBI imaging could attempt to match. 

\begin{table}[h!]
\begin{tabular}{cccc}
Telescope    & Band    & Resolution  & Area \\\hline
Chandra      & X-ray   & $0.5''-6.0''$ & $17'\times 120'$ \\
GALEX        & UV      & $5.5''      $ & 1.25$^\text{o}$ diameter \\
HST/ACS      & Optical & $0.1''      $ & $10' \times 67'$\\
HST/NICMOS   & Optical & $0.35''     $ & 0.0128 deg$^2$\\
Megacam      & Optical & $1.0''      $ & 1 deg$^2$\\
IRAC         & IR      & $2.0''      $ & $10' \times 120'$ \\
Spitzer      & IR      & $5.9''-19'' $ & $10'\times 90'$\\
VLA          & Radio   & $1.2''-4.2''$ & $30' \times 80'$
\end{tabular}
\caption{\label{table.EGS.observation}Table recapitulating the multi-wavelength coverage of the Extended Groth Strip, with observations performed as part of the AEGIS (later, CANDELS) collaborations.}
\end{table}

\pg
3C295 is a 3C\footnote{Third Cambridge Catalogue of Radio Sources, a 1959 survey of of the Northern radio sky} source, and thus one of the brighter sources in the Northern radio sky. It lies less than a degree away from the centre of the Extended Groth Strip, which has two important consequences: first, that it can potentially be used as a calibrator source for LOFAR; second, that unless well-modeled and subtracted at the observing instrument's maximum resolution, its sidelobes will pollute the EGS such that very little scientifically useful information might be recovered from a given observation. 

\section{Imaging the Full Primary Beam}
\pg
Perform DID calibration at low resolutions (no international baselines) to get complete, approximatemodel of entire EGS field

\section{Test Decorrelation}
\pg
see effect of decorrelation on LOBOS sources around EGS as function of distance from 3c295 - talk about two sources of decorrelation (direction-dependent PSF, which is modelled, and gains changing with direction, which is unmodelled and will have an impact in image). See maximum impact of gain-decorrelation: we want flat decorrelation as function of distance from 3c295.

\section{Imaging the EGS with LOFAR international stations}

\subsection{Science Goals}

\pg
if decorrelation is merciful, proceed to patchwise imaging of EGS by using the results from the sections above (DI calibration using 3c295 model, followed by subtraction of all sources seen at low-res except within the patch we want to image; image, change patch; repeat until all EGS imaged)

\newpage