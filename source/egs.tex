% !TeX spellcheck = en_UK

\section{Calibrating and Imaging the Extended Groth Strip}\label{section.calibration}
\pg
In this section, we describe the work done as part of this PhD on calibrating and imaging interferometric data on the Extended Groth Strip, an extragalactic field with rich multi-wavelength coverage. It has been long observed as part of the All-Wavelength Extended Groth Strip International Survey collaboration (\citepads{2007ApJ...660L...1D}), which later became part of the CANDELS collaboration (\citepads{2011ApJS..197...35G}). The field is centred at $\alpha=14^h17^m,\delta=+52\deg 30'$, placing it between the tail of Ursa Major and Draco. It has notably been the subject of very deep Hubble Space Telescope observations, 



talk about the field: location, size, history

talk about multi-lambda coverage: put coverage figure here!

\begin{table}[h!]
\begin{tabular}{ccccc}
Telescope    & Band   & Resolution  & Area & Reference \\\hline
Chandra      & X-ray  & 0.5''-6.0'' & $17'\times 120'$ & \\
\end{tabular}
\caption{\label{table.LOBOS.sources}Table recapitulating the positions of all 8 chosen calibrator sources, along with their distance from the observation phase centre (EGS phase centre) and calibrator phase centre (3C295), respectively.}
\end{table}


\subsection{High-resolution imaging: calibrating the International Baselines}