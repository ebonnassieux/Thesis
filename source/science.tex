% !TeX spellcheck = en_UK

\section{Radio Galaxy Classification}\label{section.science}
\pg
let's talk about radio galaxy types.

\subsection{AGN: HERG vs LERG}
\pg
AGN are galaxies with Active Galactic Nuclei, i.e. galaxies with a non-quiescent black hole at their centre. They can be further classified into two categories: HERGs and LERGS.

\subsubsection{HERGs}
\pg
HERGs are High-Excitation Radio Galaxies. They consist of those galaxies whose central SMBH show efficient accretion (of the order of 1-10\% of the Eddington rate, which is the highest rate of accretion achievable). They are also referred to as "quasar-mode" radio galaxies. HERGs become the dominant population within AGN samples at luminosities of around $10^{26}$ W/Hz at 1.4GHz. They are usually of lower stellar mass, with lower black hole masses, bluer colour, lower concentration indices and younger stellar populations (less pronounced spectral breaks at 4000 angstrom).

\pg
It would seem that HERGs are fed through radiatively efficient standard accretion disks by cold gas. These disks would be optically thick and geometrically thin.

\subsubsection{LERGs}

\pg
LERGs are Low-Excitation Radio Galaxies. They consists of those galaxies with inefficient accretion (less than a percent of the Eddington rate), and are more common 