% !TeX spellcheck = en_UK

\section{Imaging with Radio Interferometric Data}\label{section.imaging}

\pg
In this section, we will cover the difficulties introduced by the combination of sparse $uv$-coverage and weak \emph{a priori} constraints on the sky brightness distribution. While we do not need to strictly base ourselves on the RIME formalism described in \cref{section.RIME}, it remains the conceptual framework we will assume that the reader uses. For the remainder of this section, we will assume that calibration has been successfully carried out, and that we are working on the \emph{corrected visibilities}, i.e. gain-corrected visibilities. 

\pg
There exists a wide variety of methods used in imaging, from the venerable CLEAN algorithm\footnote{For an excellent beginner's introduction to CLEAN, the author heartily recommends \url{https://www.cv.nrao.edu/~abridle/deconvol/node7.html}} to cutting-edge compressed sensing and subspace deconvolution methods. We will begin this section by introducing a mathematical framework in which the problem of deconvolution can be understood, and proceed, from there, to discuss some of the deconvolution algorithms which can be used.

\pg
The formalism used in this section is based, in part or in whole, on that used by Cyril Tasse in [DDF paper].

\subsection{The Forward Operation}

\pg
Let us begin by explicitly writing out what a corrected visibility corresponds to. What we are interested in, as scientists, is the \emph{true, uncorrupted} signal reaching our instruments from astronomical sources. Assuming calibration is carried out successfully, we then have:
\begin{align}
fuck
\end{align}